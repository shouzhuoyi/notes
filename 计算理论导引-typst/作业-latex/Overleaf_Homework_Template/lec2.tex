\documentclass{homework}
\usepackage{amsmath}
\usepackage{ctex}
\author{易守拙\ 2024300001103}
\class{计算理论导引}
\date{\today}
\title{计算理论导引第二周作业}
\address{使用了这个模版:\url{https://github.com/simurgh9/hw}}

\graphicspath{{./media/}}

\begin{document} \maketitle

\question \textbf{设 $T(n), T'(n)$ 为时间可构造.证明 $T(n) + T'(n)$、$T(n) \cdot T'(n)$、$T(n)^{T'(n)}$ 均为时间可构造.能证明 $T(n)/T'(n)$ 是时间可构造吗?$\log T(n)$ 呢?}

\begin{sol}
	设 $\mathbb{M_T}$ 和 $\mathbb{M_{T'}}$ 分别是计算 $T(n)$ 和 $T'(n)$ 的图灵机.
\begin{itemize}
	\item $T(n) + T'(n)$:考察这样的$\mathbb{M_+},$它先和 $\mathbb{M_T}$执行一样的运算操作,在执行完成后,它不清空带子,而是继续进行和$\mathbb{M_{T'}}$ 一样的操作,最后把他们的计算结果加起来,这个二进制加法操作的开销是$O(\log(n))$的.所以总运行时间是 $O(T(n) + T'(n))$ 步. 因此, $T(n) + T'(n)$ 是时间可构造的.
	\item $T(n) \cdot T'(n)$:考察这样的$\mathbb{M_\cdot},$它进行如下操作:先和 $\mathbb{M_T}$执行一样的运算操作,待完成后,再一个独立的带子上写下一个1,这个循环持续$T'(n)$遍.由于$T'(n)$时间可构造,这个操作是可以被图灵机实现的,所以$T(n) \cdot T'(n)$ 是时间可构造的.
	\item $T(n)^{T'(n)}$:类似上一个,构建一个循环.\\
	 先计算 $T(n)$ (需要 $T(n)$ 步),再计算 $T'(n)$ (需要 $T'(n)$ 步);因为根据定义,我们需要最后$T(n)^{T'(n)}$ 的二进制表示,所以需要执行一个 $T'(n)$ 次的循环, 在循环体中, 执行一个 $T(n)$ 次的乘法操作.乘法操作参考(2),是时间可构造的.
	 \item $T(n)/T'(n)$:不行.若$T(n) =2^{2^n},T'(n) = 2^{n}$,需要用 $\mathbb{M_T}$ ,先算出$T(n)$的值是多少,但是$T(n) \gg T(n)/T'(n)$,不能被写成大O的形式.
	 \item $\log T(n)$:同理不行.
\end{itemize}
  
\end{sol}
\question \textbf{在通用图灵机的证明中,我们让 $|R_i| = 2 \cdot 2^{i^2}$.证明在哪一步会出问题?如果没有问题的话,我们会得到一个更高效的通用图灵机!}

\begin{sol}
	由于更改,我们必须在原来的证明中把冗余信息的存放方式修正为:要么全空,要么存$\log(\frac{R_i}{2})/2$(替代半满),要么存$\log(\frac{R_i}{2})$(替代全满).
	
	问题出在最后的移动冗余信息中.考察一个简单的图灵机 $\mathbb{M}$, 它只是简单地将其读写头向右移动 $T(n)$ 个单元格, 每步都在纸带上写入一个符号.令 $S(i)$ 为从中心点 0 到段 $R_{i-1}$ 末尾的总容量.
	$S(i) = \sum_{j=0}^{i-1} |R_j| = \sum_{j=0}^{i-1} 2 \cdot 2^{j^2}$.
	
	为了让 $M$ 的读写头移动到第 $S(k)+1$ 个单元格, $U$ 必须已经执行了所有 $i=1, 2, \dots, k$ 的重组操作,把$R_i$的冗余信息搬到前面$1\sim i-1$个格子内.则:
	$$T^‘(n) \ge \sum_{i=1}^{k} C_i = \sum_{i=1}^{k} O(2^{i^2})= O(2^{k^2})$$
	
	取$k = \sqrt{\log_2 T(n)} + 1$,代回 $T_U(n)$:
		\begin{equation}
					T^{'} (n) =O(2^{\log_2 T(n) + 2\sqrt{\log_2 T(n)} + 1})\\
			= O(2^{\log_2 T(n)} \cdot 2^{2\sqrt{\log_2 T(n)}} \cdot 2^1)\\
			= O(T(n) \cdot 2^{2\sqrt{\log_2 T(n)}})\\
		\end{equation}
他的效率很低下.

\end{sol}

\question \textbf{说明:若忽略不终止性,在第 28 页上定义的非确定的“通用”图灵机是正确的.}
\begin{sol}

	
	\subsection*{($\Rightarrow$) }
	根据非确定性计算的定义, 如果 $\mathbb{N}_\alpha$ 接受输入 $x$, 那么必然存在至少一条从初始格局到接受格局的合法计算路径.设这条接受路径为格局序列 $\mathcal{C} = (C_0, C_1, \dots, C_m)$, 其对应的非确定性选择序列为 $\mathcal{D} = (\delta_1, \dots, \delta_m)$.
	
 $\mathbb{V}$ 本身是一台非确定图灵机, 它能够探索所有可能的计算分支. 因此, 必然存在一$\mathbb{V}$的计算路径, 在该路径上, 它所猜测的格局序列恰好是 $\mathcal{C}$, 猜测的选择序列恰好是 $\mathcal{D}$. 此时被猜测的序列本身就是一条合法的接受路径, 所有验证步骤都会成功,$\mathbb{V}$将停机并成功模拟了运算.
			
	\subsection*{($\Leftarrow$) }
	如果$\mathbb{V}$ 接受输入 $\langle \alpha, x \rangle$, 那么必然存在至少一条$\mathbb{V}$的计算路径使其进入接受状态.
在这条使 $\mathbb{V}$ 接受的路径上, 它必须成功地完成“猜测”和“验证”两个阶段.$\mathbb{V}$所猜测的格局序列 $\mathcal{C} = (C_0, C_1, \dots, C_m)$ 和选择序列 $\mathcal{D} = (\delta_1, \dots, \delta_m)$ 必须满足以下所有条件:
		\begin{itemize}
			\item  $C_0$ 被验证为是 $\mathbb{N}_\alpha$ 在输入 $x$ 上的初始格局.
			\item  $C_m$ 被验证为是一个接受格局.
			\item  对所有 $i \in \{0, \dots, m-1\}$, 转移 $C_i \to C_{i+1}$ 被验证为是 $\mathbb{N}_\alpha$ 在选择 $\delta_{i+1}$ 下的一步合法操作.
		\end{itemize}
根据定义,$\mathbb{N}_\alpha$  接受 $x$.


\end{sol}

\question \textbf{证明:设图灵机 $\mathbb{M}$ 在 $T(n) = \omega(n)$步内判定 $L$.对任意 $\varepsilon > 0$,有图灵机 $\mathbb{M}'$,$\mathbb{M}'$ 能在 $\varepsilon T(n)$ 步内判定 $L$.}


\begin{sol}
	根据课堂上讲过的($T(n) = \omega(n)$不一定成立的)线性加速定理,我们知道,有图灵机$\mathbb{M}'$,它能在$F(n)\stackrel{\Delta}{=}\varepsilon_1 T(n) + n + 2 $步内判定$L$,其中$\varepsilon_1 > 0$.\\
	由于$T(n) = \omega(n)$,知$\forall \delta > 0 , \exists N \in \mathbb{N}, \forall n > N , $
	\[
	 0 <	\frac{n+2}{T(n)} < \delta\\
	 \Longrightarrow  F(n) < (\varepsilon_1 + \delta)T(n).
	\]
	我们取$\varepsilon_1 = \delta = \frac{\varepsilon}{2}$即可.
	
\end{sol}


\end{document}
